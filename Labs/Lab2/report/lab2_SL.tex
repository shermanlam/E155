\documentclass[11pt]{article}

\usepackage{verbatim}
\usepackage{amsmath}
\usepackage{amssymb}
%\usepackage{psfrag}
\usepackage{setspace}
\usepackage[top=1in, bottom=1in, left=1.25in, right=1.25in]{geometry}
%\usepackage{fancyhdr}
\usepackage{subfigure}
\usepackage{graphicx}
\usepackage{cite}
\usepackage[squaren]{SIunits}

\usepackage{listings}

\usepackage{subfloat}
%\usepackage{hyperref}

%\pagestyle{fancy}
%\lhead{E155, Lab1}
%\chead{\today}
%\rhead{Sherman Lam}
%\renewcommand{\headrulewidth}{0.4pt}
%\lfoot{}
%\cfoot{}
%\rfoot{}


\setlength{\parindent}{0pt} 	% remove the silly paragraph indents

\begin{document}



% ---------------------------------------
% Name section
% ---------------------------------------
\begin{flushleft}
Sherman Lam
\\E155
\\ \today
\end{flushleft}


% ---------------------------------------
% Title
% ---------------------------------------
\begin{center}
\begin{Large}
\textbf{Lab 2 Report: Time-Multiplexed 7-Segment Displays}
\end{Large}
\end{center}


% ---------------------------------------
% Start report
% ---------------------------------------


\section{Introduction}
\label{sec:intro}

In this lab I displayed two 7-segment displays while only using one 7-segment decoder. This is was done by time-multiplexing the two displays. With this technique, one switch is read, decoded, and shown on its corresponding 7-segment display. Then, then the second switch is read, decoded, and shown on its corresponding 7-segment display. Only one 7-segment display is on at a time. The FPGA rapidly switches between the two sets of switches and displays at a rate fasts enough to appear steady to the human eye. In addition, the sum of the two switches is displayed on the LED bar. \\

Figure \ref{fig:board}, shows the result of the lab. The DIP switches on the breadboard control the left 7-segment display (closer to the top of the breadboard) and the DIP switches on the $\mu$Mudd control the right 7-segment display (closer to the bottom of the breadboard). 

\begin{figure}[h!]
\centering
\includegraphics[scale=0.11]{board.jpg}
\caption{Assembled board.}
\label{fig:board}
\end{figure} 


\section{Design and Testing Methodology}

Notes on pin mapping:

switch 1:
s1[0] = P54
s1[1] = P55
s1[2] = P53
s1[3] = P24

switch 2:
s2[0] = P44
s2[1] = P49
s2[2] = P50
s2[3] = P51

display enable:
on1 = P87
on2 = P86

clk = P88
reset = P60

seg[0] = a = P2
seg[1] = b = P1
seg[2] = c = P4
seg[3] = d = P10
seg[4] = e = P11
seg[5] = f = P3
seg[6] = g = P7

led[0] = P28		(LSB)
led[1] = P30
led[2] = P31
led[3] = P32
led[4] = P33



Other notes:
Drop between base and emitter was measured to be 0.74-0.76 V. 0.7V approx okay.
Drop between emitter and collector was 0.12-0.15V for 5.6k base resistor. 0.09V for 2.2k resistor. 0.05V for 1k resistor. 
Drop across 220 ohm resistors is 1V

reset switch works.


\subsection{Hardware}


A PNP transistor (2N3906) was chosen for toggling the displays because the switch was acting on the anode. If the transistor's based voltage ($V_B$) is brought close to GND, the PNP would turn on. If $V_B$ is brought close to VCC, the transistor would turn off. However, since BJT transistors are current controlled, a resistor must be placed between $V_B$ and its signal pin. To select a value for this resistor, I first considered the largest current draw expected ($I_C$).

\begin{align*}
I_{C} &= (\# segments)*(I_{1segment}) \\
I_{C} &= (\# segments)*(\frac{Vcc-V_{led}}{R}) \\
I_{C} &= (\# segments)*(\frac{3.3V-1.7V}{220\Omega}) \\
I_{C} &= 7*(7.3mA) \\
I_{C} &= 51mA
\end{align*}

Then, I considered how much base current ($I_{B}$) I need in order to supply $I_{C}$. The transistor DC current gain (denoted $\beta$ or $h_{FE}$) was approximated from Figure \ref{fig:2N3906_gain} to be 120. So, $I_{B}$ was found to be:

\begin{align*}
I_{B} &= \frac{I_{C}}{\beta} \\
I_{B} &= \frac{51mA}{120} \\
I_{B} &= 0.4mA
\end{align*}

We can then find the resistor need to supply this voltage when the signal pin is pulled LOW. Because this is a silicon diode, $V_{EB}=0.7V$.  Note that R was rounded to the nearest resistor value available in the lab. This is okay because the the value of R is not critical. It simply needs to be high enough to prevent the transistor from burning out (if $V_EB$ is too high) and low enough to allow sufficient current to light up the LEDs. If the value of the R is lower than ideal, the transistor will saturate but still function.

\begin{align*}
R &= \frac{Vcc-V_{EB}}{I_{B}} \\
R &= \frac{3.3-0.7}{0.4mA} \\
R &= 6.1k\Omega	\\
R &\approx 5.6k\Omega
\end{align*}



\begin{figure}[h!]
\centering
\includegraphics[scale=1]{2N3906_gain.png}
\caption{DC gain of 2N3906 transistor. Image obtained from 2N3906 datasheet.}
\label{fig:2N3906_gain}
\end{figure} 





PNP transistors were also used because MOSFETs were not available. 

\subsection{Software}




\begin{table}[h]
\centering
\begin{tabular}{|c|c|c|c|c|c|c|c|c|c|c|c|c|}
\hline
\multicolumn{13}{|c|}{7-Segment Display Truth Table}                                  \\ \hline
\multicolumn{5}{|c|}{Inputs}                      & \multicolumn{8}{c|}{Ouputs}       \\ \hline
s{[}3{]} & s{[}2{]} & s{[}1{]} & s{[}0{]} & (hex) & G & F & E & D & C & B & A & (hex) \\ \hline
0        & 0        & 0        & 0        & 0x0   & 1 & 0 & 0 & 0 & 0 & 0 & 0 & 0x40  \\ \hline
0        & 0        & 0        & 1        & 0x1   & 1 & 1 & 1 & 1 & 0 & 0 & 1 & 0x79  \\ \hline
0        & 0        & 1        & 0        & 0x2   & 0 & 1 & 0 & 0 & 1 & 0 & 0 & 0x24  \\ \hline
0        & 0        & 1        & 1        & 0x3   & 0 & 1 & 1 & 0 & 0 & 0 & 0 & 0x30  \\ \hline
0        & 1        & 0        & 0        & 0x4   & 0 & 0 & 1 & 1 & 0 & 0 & 1 & 0x19  \\ \hline
0        & 1        & 0        & 1        & 0x5   & 0 & 0 & 1 & 0 & 0 & 1 & 0 & 0x12  \\ \hline
0        & 1        & 1        & 0        & 0x6   & 0 & 0 & 0 & 0 & 0 & 1 & 0 & 0x02  \\ \hline
0        & 1        & 1        & 1        & 0x7   & 1 & 1 & 1 & 1 & 0 & 0 & 0 & 0x78  \\ \hline
1        & 0        & 0        & 0        & 0x8   & 0 & 0 & 0 & 0 & 0 & 0 & 0 & 0x00  \\ \hline
1        & 0        & 0        & 1        & 0x9   & 0 & 0 & 1 & 1 & 0 & 0 & 0 & 0x18  \\ \hline
1        & 0        & 1        & 0        & 0xA   & 0 & 0 & 0 & 1 & 0 & 0 & 0 & 0x08  \\ \hline
1        & 0        & 1        & 1        & 0xB   & 0 & 0 & 0 & 0 & 0 & 1 & 1 & 0x03  \\ \hline
1        & 1        & 0        & 0        & 0xC   & 0 & 1 & 0 & 0 & 1 & 1 & 1 & 0x27  \\ \hline
1        & 1        & 0        & 1        & 0xD   & 0 & 1 & 0 & 0 & 0 & 0 & 1 & 0x21  \\ \hline
1        & 1        & 1        & 0        & 0xE   & 0 & 0 & 0 & 0 & 1 & 1 & 0 & 0x06  \\ \hline
1        & 1        & 1        & 1        & 0xF   & 0 & 0 & 0 & 1 & 1 & 1 & 0 & 0x0E  \\ \hline
\end{tabular}
\caption{Truth table for 7-Segment LED decoder}
\label{table:7seg_decoder}
\end{table}




\label{sec:software_LEDbar}

\clearpage


\subsubsection{Simulation}

The code's logic was tested in ModelSim-Altera. The following show the results of the wave simulations that were run.

\begin{figure}[h!]
\centering
\includegraphics[scale=0.65]{clk_works.png}
\caption{The counter increments at every clock cycle.}
\label{fig:wave_clk}
\end{figure} 


\begin{figure}[h!]
\centering
\includegraphics[scale=0.7]{mux_all.png}
\caption{Logic for 7-segment decoder.}
\label{fig:wave_mux}
\end{figure} 


\begin{figure}[h!]
\centering
\includegraphics[scale=0.75]{no_reset.png}
\caption{If the reset signal is never HIGH, the control signals for toggling the 7-segment displays (on1 and on2) will not resolve to a logic level.}
\label{fig:wave_reset_no}
\end{figure} 


\begin{figure}[h!]
\centering
\includegraphics[scale=0.8]{yes_reset.png}
\caption{Once the reset signal is HIGH, the control signals for toggling the 7-segment displays (on1 and on2) are initialized to hard-coded starting values.}
\label{fig:wave_reset_yes}
\end{figure}


\begin{figure}[h!]
\centering
\includegraphics[scale=1]{toggling.png}
\caption{When on1 and on2 will swap states, s3 (the signal used for the 7-segment decoding) changes from s1 to s2 or vice versa.}
\label{fig:wave_toggle}
\end{figure}


\begin{figure}[h!]
\centering
\includegraphics[scale=1]{sum.png}
\caption{The summation of the two input switch values (s1 and s2) is sent to the led bar (led[4:0]).}
\label{fig:wave_sum}
\end{figure}


\clearpage

\section{Technical Documentation}


\subsection{7-segment Displays Schematic}

\begin{figure}[h!]
\centering
\includegraphics[scale=0.7, angle=90]{seven_segment_all.png}
\caption{Full schematic for dual 7-segment display. Note that on1 and on2 toggle the two displays on and off. Only one or the other is on at any given time.}
\label{fig:seven_seg_sch}
\end{figure} 


\begin{figure}[h!]
\centering
\includegraphics[scale=0.54]{reset.png}
\caption{Schematic for reset button.}
\label{fig:reset_sch}
\end{figure} 


\begin{figure}[h!]
\centering
\includegraphics[scale=0.54]{s1.png}
\caption{Schematic for DIP switch 1. This controls display segment 1 (left).}
\label{fig:s1_sch}
\end{figure} 


\begin{figure}[h!]
\centering
\includegraphics[scale=0.54]{s2.png}
\caption{Schematic for DIP switch 2. This controls display segment 2 (right).}
\label{fig:s2_sch}
\end{figure} 


\clearpage
\subsection{System Verilog Code}

% \small\begin{verbatim}
\begin{lstlisting}[language=Verilog,numbers=left,basicstyle=\footnotesize]

/* This is the main module. It selects which set of switch
   outputs to use and then decodes the number of the selected
   switch. This also sets the clock that time-multiplexes the 
   two 7 segment outputs.
   
   Author: Sherman Lam
   Email: slam@g.hmc.edu
   Date: Sep 17, 2014
*/
module lab2_SL(input logic clk, reset,
               input logic [3:0] s1,s2, //DIP switches
               output logic on1, on2,   //if on1 is pulled LOW, LED set 1 is on.
               output logic [6:0] seg,
               output logic [4:0] led); //segment states    
   
   // time multiplexing
   multiplexer m1(.clk(clk), .on1(on1), .reset(reset));
   
   // the segments always have opposite states.
   assign on2 = ~on1;      
   
   // select the right set of switches.
   // on1 -> s1 is used. on2 -> s2 is used
   // if on1 is pulled LOW, LED set 1 is on.
   logic [3:0] s3;
   assign s3 = on1? s2 : s1;  
   
   // 7 segment decoder
   led7Decoder decoder(.s(s3), .seg(seg));
   
   // sum the outputs and write to LED bar
   assign led = s1 + s2;
   
   
endmodule


/* This module time multiplexes

   Author: Sherman Lam
   Email: slam@g.hmc.edu
   Date: Sep 17, 2014
*/
module multiplexer(  input logic clk, reset,
                     output logic on1);
   // time multiplexer for switching bewteen displays
   logic [18:0] hPeriod = 19'd333333;  // 120Hz toggling
   logic [18:0] counter = 'b0;
      
   always_ff @(posedge clk, posedge reset) begin
      if (reset)     
         on1 = 1'b0;
      else begin
         if (counter >= hPeriod) begin
            counter = 'b0;
            on1 = ~on1;
         end
         else
            //on1 = on1;
            counter <= counter + 1'b1;
      end
   end
   
endmodule


/* This module decodes the switch inputs into an output for the 
   7 segment display on the development board.
   s[3:0] = [sw3, ... ,sw1]
   seg[6:0] = [g,f, ... ,b,a]
   
   Author: Sherman
   Email: slam@g.hmc.edu
   Date: Sep 9, 2014
*/
module led7Decoder(  input logic [3:0] s,       //4 DIP switches
                     output logic [6:0] seg);   //segments in 7-seg display
                     
   always_comb begin
      //lookup table for s-seg relationship
      case(s)
         4'h0: seg = 7'b100_0000;      // 0x0
         4'h1: seg = 7'b111_1001;      // 0x1
         4'h2: seg = 7'b010_0100;      // 0x2
         4'h3: seg = 7'b011_0000;      // 0x3
         4'h4: seg = 7'b001_1001;      // 0x4
         4'h5: seg = 7'b001_0010;      // 0x5
         4'h6: seg = 7'b000_0010;      // 0x6
         4'h7: seg = 7'b111_1000;      // 0x7
         4'h8: seg = 7'b000_0000;      // 0x8
         4'h9: seg = 7'b001_1000;      // 0x9
         4'ha: seg = 7'b000_1000;      // 0xA
         4'hb: seg = 7'b000_0011;      // 0xB
         4'hc: seg = 7'b010_0111;      // 0xC
         4'hd: seg = 7'b010_0001;      // 0xD
         4'he: seg = 7'b000_0110;      // 0xE
         4'hf: seg = 7'b000_1110;      // 0xF
         default: seg = 7'b111_1110;      // default to a dash
      endcase
      
   end
endmodule



\end{lstlisting}
% \end{verbatim}


\clearpage

\section{Results and Discussion}

The system works as expected. The number set by the DIP switches on the breadboard is displayed on the left 7-segment display. The number set by the DIP switches on the $\mu$Mudd is displayed on the right 7-segment display. The FPGA flashes both displays at 60Hz. This flashing is unnoticeable when viewed by the human eye. Since each display is operating at 50$\%$ duty cycle, the intensity of the display is about half that of a display being held on (100$\%$ duty cycle). In addition, when the reset button is pressed, display 1 (left) turns on and display 2 (right) turns off. This states is held until the reset button is released.  \\


\section{Conclusion}

\subsection{Time Spent}

\begin{description}
	\item[Programming, Simulating] 2.5hrs
	\item[Breadboarding] 2hrs
	\item[Writing Report] 3hrs
	\item[Total Time Spent] 7.5hrs
\end{description}

\subsection{Suggestions for lab}

The current lab asks the student to use the DIP switches on the $\mu$Mudd in addition to 4 wires to control the two 7-segment displays. However, many people were asking if 4 wires meant an extra set of DIP switch. The instructions seem confusing in regards to this detail. Unless the lab's goal is to purposefully give semi-ambiguous instructions in order to force students to think (channeling E80, eh?), I would suggest explicitly give instructions to use a second set of DIP switches.


\end{document}

