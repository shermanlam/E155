\documentclass[11pt]{article}

\usepackage{verbatim}
\usepackage{amsmath}
\usepackage{amssymb}
%\usepackage{psfrag}
\usepackage{setspace}
\usepackage[top=1in, bottom=1in, left=1.25in, right=1.25in]{geometry}
%\usepackage{fancyhdr}
\usepackage{subfigure}
\usepackage{graphicx}
\usepackage{cite}
\usepackage[squaren]{SIunits}
\usepackage{listings}
\usepackage{subfloat}
%\usepackage{hyperref}

%\pagestyle{fancy}
%\lhead{E155, Lab1}
%\chead{\today}
%\rhead{Sherman Lam}
%\renewcommand{\headrulewidth}{0.4pt}
%\lfoot{}
%\cfoot{}
%\rfoot{}


\setlength{\parindent}{0pt} 	% remove the silly paragraph indents

\begin{document}



% ---------------------------------------
% Name section
% ---------------------------------------
\begin{flushleft}
Sherman Lam
\\E155
\\ \today
\end{flushleft}


% ---------------------------------------
% Title
% ---------------------------------------
\begin{center}
\begin{Large}
\textbf{Lab 2 Report: Multiplexed Display}
\end{Large}
\end{center}


% ---------------------------------------
% Start report
% ---------------------------------------


\section{Introduction}
\label{sec:intro}



\section{Design and Testing Methodology}

\subsection{Hardware}

Notes on pin mapping:

switch 1:
s1[0] = P54
s1[1] = P55
s1[2] = P53
s1[3] = P24

switch 2:
s2[0] = P44
s2[1] = P49
s2[2] = P50
s2[3] = P51

display enable:
on1 = P87
on2 = P86

clk = P88
reset = P60

seg[0] = a = P2
seg[1] = b = P1
seg[2] = c = P4
seg[3] = d = P10
seg[4] = e = P11
seg[5] = f = P3
seg[6] = g = P7


Other notes:
Drop between base and emitter was measured to be 0.74-0.76 V. 0.7V approx okay.
Drop between emitter and collector was 0.12-0.15V for 5.6k base resistor. 0.09V for 2.2k resistor. 0.05V for 1k resistor. 
Drop across 220 ohm resistors is 1V

reset switch works.




\subsection{Software}



\label{sec:software_LEDbar}

\subsubsection{Testing and Flashing}



\section{Technical Documentation}



\subsection{System Verilog Code}

\small\begin{verbatim}

/* This module is the wrapper for the project. It instantiates
  an instance of the led bar decoder and 7 segment decoder
  
  Author: Sherman Lam
  Email: slam@g.hmc.edu
  Date: Sep 9, 2014
*/

module lab1_SL( input logic clk,        //clock
            input logic [3:0] s,        //4 DIP switches
            output logic [7:0] led,     //8 lights on LED bar
            output logic [6:0] seg);    //segments in 7-seg display
  
  //instance of the led bar decoder
  ledBarDecoder   bar(.clk(clk), .s(s), .led(led));   
  //instance of 7-seg display decoder
  led7Decoder     led7(.s(s), .seg(seg)); 

endmodule


\end{verbatim}




\clearpage

\section{Results and Discussion}




\section{Conclusion}


Time spent: so far, about 4.5hrs

\end{document}
