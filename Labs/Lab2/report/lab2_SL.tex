\documentclass[11pt]{article}

\usepackage{verbatim}
\usepackage{amsmath}
\usepackage{amssymb}
%\usepackage{psfrag}
\usepackage{setspace}
\usepackage[top=1in, bottom=1in, left=1.25in, right=1.25in]{geometry}
%\usepackage{fancyhdr}
\usepackage{subfigure}
\usepackage{graphicx}
\usepackage{cite}
\usepackage[squaren]{SIunits}

\usepackage{listings}

\usepackage{subfloat}
%\usepackage{hyperref}

%\pagestyle{fancy}
%\lhead{E155, Lab1}
%\chead{\today}
%\rhead{Sherman Lam}
%\renewcommand{\headrulewidth}{0.4pt}
%\lfoot{}
%\cfoot{}
%\rfoot{}


\setlength{\parindent}{0pt} 	% remove the silly paragraph indents

\begin{document}



% ---------------------------------------
% Name section
% ---------------------------------------
\begin{flushleft}
Sherman Lam
\\E155
\\ \today
\end{flushleft}


% ---------------------------------------
% Title
% ---------------------------------------
\begin{center}
\begin{Large}
\textbf{Lab 2 Report: Multiplexed Display}
\end{Large}
\end{center}


% ---------------------------------------
% Start report
% ---------------------------------------


\section{Introduction}
\label{sec:intro}



\section{Design and Testing Methodology}

\subsection{Hardware}

Notes on pin mapping:

switch 1:
s1[0] = P54
s1[1] = P55
s1[2] = P53
s1[3] = P24

switch 2:
s2[0] = P44
s2[1] = P49
s2[2] = P50
s2[3] = P51

display enable:
on1 = P87
on2 = P86

clk = P88
reset = P60

seg[0] = a = P2
seg[1] = b = P1
seg[2] = c = P4
seg[3] = d = P10
seg[4] = e = P11
seg[5] = f = P3
seg[6] = g = P7

led[0] = P28		(LSB)
led[1] = P30
led[2] = P31
led[3] = P32
led[4] = P33



Other notes:
Drop between base and emitter was measured to be 0.74-0.76 V. 0.7V approx okay.
Drop between emitter and collector was 0.12-0.15V for 5.6k base resistor. 0.09V for 2.2k resistor. 0.05V for 1k resistor. 
Drop across 220 ohm resistors is 1V

reset switch works.




\subsection{Software}



\label{sec:software_LEDbar}

\subsubsection{Testing and Flashing}



\section{Technical Documentation}



\subsection{System Verilog Code}

% \small\begin{verbatim}
\begin{lstlisting}[language=Verilog,numbers=left,basicstyle=\footnotesize]
/* This is the main module. It selects which set of switch
   outputs to use and then decodes the number of the selected
   switch. This also sets the clock that time-multiplexes the 
   two 7 segment outputs.
   
   Author: Sherman Lam
   Email: slam@g.hmc.edu
   Date: Sep 17, 2014
*/
module lab2_SL(input logic clk, reset,
               input logic [3:0] s1,s2, //DIP switches
               output logic on1, on2,   //if on1 is pulled LOW, LED set 1 is on.
               output logic [6:0] seg); //segment states    
   
   // time multiplexing
   multiplexer m1(.clk(clk), .on1(on1), .reset(reset));
   
   // the segments always have opposite states.
   assign on2 = ~on1;      
   
   // select the right set of switches.
   // on1 -> s1 is used. on2 -> s2 is used
   logic [3:0] s3;
   // if on1 is pulled LOW, LED set 1 is on.
   assign s3 = on1? s2 : s1;  
   
   // 7 segment decoder
   led7Decoder decoder(.s(s3), .seg(seg));
   
endmodule


/* This module time multiplexes

   Author: Sherman Lam
   Email: slam@g.hmc.edu
   Date: Sep 17, 2014
*/
module multiplexer(  input logic clk, reset,
                     output logic on1);
   // time multiplexer for switching bewteen displays
   logic [18:0] hPeriod = 19'd333333;  // 120Hz toggling
   logic [18:0] counter = 'b0;
      
   always_ff @(posedge clk, posedge reset) begin
      if (reset)     
         on1 = 1'b0;
      else begin
         if (counter >= hPeriod) begin
            counter = 'b0;
            on1 = ~on1;
         end
         else
            //on1 = on1;
            counter <= counter + 1'b1;
      end
   end
   
endmodule


/* This module decodes the switch inputs into an output for the 
   7 segment display on the development board.
   s[3:0] = [sw3, ... ,sw1]
   seg[6:0] = [g,f, ... ,b,a]
   
   Author: Sherman
   Email: slam@g.hmc.edu
   Date: Sep 9, 2014
*/
module led7Decoder(  input logic [3:0] s,       //4 DIP switches
                     output logic [6:0] seg);   //segments in 7-seg display
                     
   always_comb begin
      //lookup table for s-seg relationship
      case(s)
         4'h0: seg = 7'b100_0000;      // 0x0
         4'h1: seg = 7'b111_1001;      // 0x1
         4'h2: seg = 7'b010_0100;      // 0x2
         4'h3: seg = 7'b011_0000;      // 0x3
         4'h4: seg = 7'b001_1001;      // 0x4
         4'h5: seg = 7'b001_0010;      // 0x5
         4'h6: seg = 7'b000_0010;      // 0x6
         4'h7: seg = 7'b111_1000;      // 0x7
         4'h8: seg = 7'b000_0000;      // 0x8
         4'h9: seg = 7'b001_1000;      // 0x9
         4'ha: seg = 7'b000_1000;      // 0xA
         4'hb: seg = 7'b000_0011;      // 0xB
         4'hc: seg = 7'b010_0111;      // 0xC
         4'hd: seg = 7'b010_0001;      // 0xD
         4'he: seg = 7'b000_0110;      // 0xE
         4'hf: seg = 7'b000_1110;      // 0xF
         default: seg = 7'b111_1110;      // default to a dash
      endcase
      
   end
endmodule

\end{lstlisting}
% \end{verbatim}


\clearpage

\section{Results and Discussion}




\section{Conclusion}


Time spent: so far, about 4.5hrs

\end{document}
