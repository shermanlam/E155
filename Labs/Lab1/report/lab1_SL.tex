\documentclass[11pt]{article}

\usepackage{verbatim}
\usepackage{amsmath}
\usepackage{amssymb}
%\usepackage{psfrag}
\usepackage{setspace}
\usepackage[top=1in, bottom=1in, left=1.25in, right=1.25in]{geometry}
%\usepackage{fancyhdr}
\usepackage{subfigure}
\usepackage{graphicx}
\usepackage{cite}
\usepackage[squaren]{SIunits}

%\pagestyle{fancy}
%\lhead{E155, Lab1}
%\chead{\today}
%\rhead{Sherman Lam}
%\renewcommand{\headrulewidth}{0.4pt}
%\lfoot{}
%\cfoot{}
%\rfoot{}


\setlength{\parindent}{0pt} 	% remove the silly paragraph indents

\begin{document}



% ---------------------------------------
% Name section
% ---------------------------------------
\begin{flushleft}
Sherman Lam
\\E155
\\ \today
\end{flushleft}


% ---------------------------------------
% Title
% ---------------------------------------
\begin{center}
\begin{Large}
\textbf{Lab 1 Report: Utility Board Assembly}
\end{Large}
\end{center}


% ---------------------------------------
% Start report
% ---------------------------------------


\section{Introduction}
\label{sec:intro}

This lab consisted of two main components: 
	\begin{enumerate} \itemsep0pt
		\item Assembling the development board
		\item Writing a simple program for interfacing LEDs and switches
	\end{enumerate}

In the first section of this lab, I assembled the E155 Utility Board from the provide kit. This board serves as a development board for a PIC32 microcontroller and a Cyclone III FPGA. The kit included basic through-hole components such as resistors, capacitors, switches, header pins, and LEDs. Other accessories included a VGA port, JTAG pins, and a jack for programming the PIC. The only surface-mount components were 3 voltage regulators (for 2.5V, 1.2V, and 3.3V). I ran some basic checks to verify that the board performed as expected. These checks included measuring the clock output and voltage regulator outputs.\\

In the second section of this lab, I used Quartus II to design an interface between the dip switches on the development board LEDs. The LEDs included a LED bar on the development board and a 7-segment display that was wired off the board.


\section{Design and Testing Methodology}

\subsection{Hardware}

The design of the development board was set by the course instructors and so it was simply assembled according to the provided lab instructions. As briefly mentioned in \ref{sec:intro} several tests were run to verify that the hardware on the board functioned according to my expectations. The results included the following:

	\begin{description} \itemsep0pt
		\item[Pre-assembly Inspection ] No visible defects were found on the PCB or the components. Using a multimeter, no short was detected across Vin and GND.
		\item[Power Supply ] When the board was powered by a 5V regulated input, the outputs of the voltage regulators read within 0.05V of the nominal voltage outputs. 
		\item[Clock ] The output of the clock measured by an oscilloscope to be within 1MHz of the nominal frequency (40MHz).
		\item[LED bar ] To check the LED bar's functionality, first the pin to one LED was connected (via a wire on the breadboard) to the pin of a switch. If the LED toggles with the switch, the LED was deemed functional. This test was repeated for all the LEDs. All LEDs performed as expected.
		\item[Switches ] To check the switches' functionality, first the pin to one LED on the LED bar was connected to the pin of one switch. If the LED toggles with the switch, the switch was deemed functional. This test was repeated for all the switches. All switches performed as expected.
		
		
	\end{description}


In designing the 7-segment LED circuit, I aimed to 
	

\subsection{Software}



\section{Technical Documentation}


\section{Results and Discussion}


\section{Conclusions}



\end{document}
