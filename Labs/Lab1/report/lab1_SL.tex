\documentclass[11pt]{article}

\usepackage{verbatim}
\usepackage{amsmath}
\usepackage{amssymb}
%\usepackage{psfrag}
\usepackage{setspace}
\usepackage[top=1in, bottom=1in, left=1.25in, right=1.25in]{geometry}
\usepackage{fancyhdr}
\usepackage{subfigure}
\usepackage{graphicx}
\usepackage{cite}
\usepackage[squaren]{SIunits}

%\pagestyle{fancy}
\lhead{E155, Lab1}
\chead{\today}
\rhead{Sherman Lam}
\renewcommand{\headrulewidth}{0.4pt}
\lfoot{}
\cfoot{}
\rfoot{}


\setlength{\parindent}{0pt} 	% remove the silly paragraph indents

\begin{document}



% ---------------------------------------
% Name section
% ---------------------------------------
\begin{flushleft}
Sherman Lam
\\E155
\\ \today
\end{flushleft}


% ---------------------------------------
% Title
% ---------------------------------------
\begin{center}
\begin{Large}
\textbf{Lab 1 Report: Utility Board Assembly}
\end{Large}
\end{center}


% ---------------------------------------
% Start report
% ---------------------------------------


\section{Introduction}

This lab consisted of two main components: 
	\begin{enumerate} \itemsep0pt
		\item Assembling the development board
		\item Writing a simple program for interfacing LEDs and switches
	\end{enumerate}

In the first section of this lab, I assembled the E155 Utility Board from the provide kit. This board serves as a development board for a PIC32 microcontroller and a Cyclone III FPGA. The kit included basic through-hole components such as resistors, capacitors, switches, header pins, and LEDs. Other accessories included a VGA port, JTAG pins, and a jack for programming the PIC. The only surface-mount components were 3 voltage regulators (for 2.5V, 1.2V, and 3.3V). I ran some basic checks to verify that the board performed as expected. These checks included measuring the clock output and voltage regulator outputs.




\section{Design and Testing Methodology}



\section{Technical Documentation}


\section{Results and Discussion}


\section{Conclusions}



\end{document}
